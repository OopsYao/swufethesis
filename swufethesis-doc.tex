%^^A 由于User Manual的内容较多,将其与swufethesis.dtx分离
%^^A 由于这个文件实际上是被以`\DocInput`的方式引入,所以其注释格式遵循dtx
\GetFileInfo{swufethesis.dtx}
\title{\color{SwufeBlue}\SwufeThesis{}:西南财经大学学位论文模板\thanks{对应
    版本\fileversion,发布日期~\filedate。}}
\date{}
\maketitle

\begin{abstract}
  该文档是西南财经大学学位论文\LaTeX{}模板(非官方)\SwufeThesis{}的使用说明文档,
  用户可以根据自己的需求来查阅该手册。
\end{abstract}

\clearpage
\tableofcontents

\clearpage
\section{模板介绍}
\SwufeThesis\ (\textbf{S}outh\textbf{w}estern \textbf{U}niversity
of \textbf{F}inance and \textbf{E}conomics \LaTeX\ \textbf{Thesis} Template)
是为帮助西南财经大学毕业生更简易高效地撰写毕业论文而编写的\LaTeX{}论文模板。

\section{使用说明}
文档类\cls{swufethesis}提供了\cs{swufesetup}进行相关的信息设定,同时也提供了
其他的一些宏和环境,将在下面一一介绍。

\subsection{论文选项}
\DescribeMacro{\swufesetup}
\cs{swufesetup}命令是最主要的控制文档选项的方式,按如下键值对的方式使用。
每个键值中由半角等号(|=|)分隔键值,同时用半角逗号(|,|)分隔不同的键值对。
但注意到\cs{swufesetup}命令中不要包含空行。
\begin{latex}
  \swufesetup{
    key1 = value1,
    key2 = {a value, with comma}
  }
\end{latex}

\subsubsection{输出格式}
\DescribeOption{output}
可选项有 |print|与 |check|,分别表示打印版和检测版,默认为 |print|。
打印版会输出封面、承诺页、封底等页面,且保证章节页面都是右开的,从而
会产生一些必要的对打印友好的空白页;
检测版将去除封面、承诺页和封底。
\begin{latex}
  \swufesetup{
    output = print
  }
\end{latex}

\subsubsection{学位}
\DescribeOption{degree}
设置论文的学位,可选值有 |bachelor|、|master|、|doctor|,分别对应学士、硕士和博士
学位论文,默认为 |bachelor|。
\begin{latex}
  \swufesetup{
    degree = bachelor
  }
\end{latex}

\subsubsection{学位类型}
\DescribeOption{degree-type}
研究生学位类型,可选值有 |academic|、|professional|、|equivalent|,分别对应学术型、
专业型和同等学力学位,默认为 |academic|。
\begin{latex}
  \swufesetup{
    degree-type = academic
  }
\end{latex}

\subsubsection{标题}
\DescribeOption{title}
\DescribeOption{title*}
中英文标题,本科生只需设置中文标题 |title|即可。
\begin{latex}
  \swufesetup{
    title = 中文标题,
    title* = English
  }
\end{latex}

\subsubsection{专业}
\DescribeOption{discipline}
设置本科生封面专业和研究生学科专业字段。
\begin{latex}
  \swufesetup{
    discipline = 计算机科学与技术(金融信息化方向),
  }
\end{latex}

\subsubsection{学号}
\DescribeOption{student-id}
对于同等学力研究生论文此项不生效。
\begin{latex}
  \swufesetup{
    student-id = 41799999
  }
\end{latex}

\subsubsection{作者姓名}
\DescribeOption{author}
姓名。
\begin{latex}
  \swufesetup{
    author = 我能写完的
  }
\end{latex}

\subsubsection{指导老师}
\DescribeOption{supervisor}
指导老师。
\begin{latex}
  \swufesetup{
    supervisor = XX教授
  }
\end{latex}

\subsubsection{成文日期}
\DescribeOption{date}
成文日期,需为 |YYYY-MM-DD|格式,同时也用于承诺页上的签署日期。默认是当前日期。
\begin{latex}
  \swufesetup{
    date = 2021-01-04
  }
\end{latex}

\subsubsection{学院}
\DescribeOption{school}
本科生毕业论文的学院。
\begin{latex}
  \swufesetup{
    school = 经济信息工程学院
  }
\end{latex}

\subsubsection{届数}
\DescribeOption{index}
本科生毕业论文的届数。默认根据成文日期设定,成文日期当年的9月(含)至次年的9月
(不含)视为同一届,如成文日期为 |2021-01-04|时则对应届数为2017。
\begin{latex}
  \swufesetup{
    index = 2017
  }
\end{latex}

\subsubsection{密级}
\DescribeOption{secret-level}
\DescribeOption{secret-year}
研究生学位论文的密级与保密年限。对于不保密的论文可以不用设置 |secret-year|。
\begin{latex}
  \swufesetup{
    secret-level = 保密,
    secret-year = 10
  }
\end{latex}

\subsubsection{分类号}
\DescribeOption{udc}
\DescribeOption{clc}
研究生论文的分类号,|udc|和 |clc|分别指UDC分类号和中图分类号。
\begin{latex}
  \swufesetup{
    udc = 004.02,
    clc = TP301.6
  }
\end{latex}

\subsection{封面与学术申明}
\DescribeMacro{\maketitle}
\DescribeMacro{\statement}
封面与学术申明分别由\cs{maketitle}和\cs{statement}生成。

\subsection{前言部分}
\DescribeMacro{\frontmatter}
前言部分由\cs{frontmatter}开启,这部分应该包括中英文摘要和目录。

\subsubsection{摘要}
\DescribeEnv{abstract}
\DescribeEnv{abstract*}
\env{abstract}和\env{abstract*}环境分别表示中英文摘要,同时接受一个必选参数,
里面是对应的中英文关键词,不同的关键词之间用半角逗号(|,|)分隔。
\begin{center}
|\begin{abstract}|\marg{keywords}, |\begin{abstract*}|\marg{keywords}
\end{center}
\begin{latex}
  \begin{abstract}{关键词1, 关键词2, 关键词3}
    简短的摘要。
  \end{abstract}

  \begin{abstract*}{keyword1, keyword2, keyword3}
    Brief abstract.
  \end{abstract*}
\end{latex}

\subsubsection{目录}
\DescribeMacro{\tableofcontents}
目录由\cs{tableofcontents}生成。

\subsection{正文部分}
\DescribeMacro{\mainmatter}
正文部分由\cs{mainmatter}开启,每一章应由\cs{chapter}设定。

\subsection{参考文献}
目前\cls{swufethesis}仅支持通过\pkg{biblatex}设置参考文献。
所以应先通过\cs{addbibiresource}\marg{bib file}设置
引用文献库,并在文中通过\cs{cite}\marg{entry}来引用,最后通过
\cs{printbibliography}打印参考文献。

\subsection{附录}
\DescribeMacro{\appendix}
附录由\cs{appendix}开启。

\subsection{其他章节}
\DescribeMacro{\backmatter}
\cs{backmatter}标记附录的结束。这之后的\cs{chapter}命令将不会被编号,但会被纳入到目录中。

\subsubsection{致谢}
\DescribeEnv{acknowledgments}
直接在\env{acknowledgments}环境里写感谢的话就好了。
\begin{latex}
  \begin{acknowledgments}
    ……
    还要感谢\SwufeThesis{}节省了论文的排版时间!
  \end{acknowledgments}
\end{latex}

\subsection{杂项}
\DescribeMacro{\SwufeThesis}
\DescribeMacro{\swufethesis}
\DescribeMacro{\swufeversion}
\DescribeMacro{\swufedate}
\cs{SwufeThesis}或者\cs{swufethesis}会打印项目标识\SwufeThesis{},
而 \cs{swufeversion}与 \cs{swufedate}会打印当前的版本和发布日期(YYYY/MM/DD)。

\section{致谢}
这个项目在开发的过程借鉴了许多 \href{https://github.com/tuna/thuthesis}{tuna/thuthesis}
的设计,也参考了 \href{https://github.com/stone-zeng/fduthesis}{stone-zeng/fduthesis}、
\href{https://github.com/sukanka/LaTeX-SWUFE-Bachelor-Thesis}{sukanka/LaTeX-SWUFE-Bachelor-Thesis}
等项目,在这里感谢他们的工作!
