% \iffalse
%<*driver>
\ProvidesFile{swufedoc.dtx}
%</driver>
%<class>\NeedsTeXFormat{LaTeX2e}[1999/12/01]
%<class>\ProvidesClass{swufedoc}
[2020/02/22 1.0.0-alpha.1 SwufeThesis Doc Template]
%
%<*driver>
\documentclass{swufedoc}
\begin{document}
  \DocInput{\jobname.dtx}
\end{document}
%</driver>
% \fi
% \title{swufedoc}
% \maketitle
% \begin{abstract}
%   \cls{swufedoc}是\pkg{swufethesis}撰写文档时内部使用的文档类。
% \end{abstract}
% \tableofcontents
% \section{简介}
% \cls{swufedoc}用于\pkg{swufethesis}文档编写,主要进行了一些格式上的设定。
% \DescribeMacro{\SwufeThesis}
% 同时提供有宏\cs{SwufeThesis}可以打印这个项目的标识\SwufeThesis{}。
% \section{实现细节}
% \subsection{装载类与宏包}
% \cls{swufedoc}基于\textsf{l3doc}文档类,
% 目前的问题是\env{macrocode}环境中的中文字体不遵循\cs{setCJKmonofont}的设定。
% 现在只能尽量减少\env{macrocode}中的中文注释。
% 如果使用\cls{ltxdoc}文档类,甚至连\env{verbatim}环境也会出现上面的问题。
%
% 值得一提的是随\pkg{ctex}宏集一起发行的\cls{ctxdoc}文档类,
% 该类中\env{macrocode}环境的中文字体设定表现正常,但是并没有正式文档(或许是旨在
% 对内使用?),从而也不方便进行更进一步的定制。
% \footnote{事实上\cls{l3doc}也是实验性的,也没有正式发布文档,但提供了 |dtx|文档可供编译。}
%    \begin{macrocode}
\LoadClass{l3doc}
\RequirePackage[fontset=none]{ctex}
\RequirePackage{fontspec}
%    \end{macrocode}
% \subsection{文档信息相关}
% 提供一些关于描述\SwufeThesis{}本身的命令。
% \begin{macro}{\SwufeThesis}
% \cs{SwufeThesis}会打印“\SwufeThesis”,可以作为本项目的标识。
%    \begin{macrocode}
\def\SwufeThesis{\textsc{Swufe\-Thesis}}
%    \end{macrocode}
% \end{macro}
% \subsection{字体设定}
% 中文使用思源字体系列,西文使用\TeX\ Gyre系列中的Pagella和Heros。
% 如果没有安装思源字体,则会使用\pkg{fandol}字体代替。
%    \begin{macrocode}
\newcommand{\swufe@set@font@pagella}{%
  \setmainfont{texgyrepagella}[
    Extension = .otf,
    UprightFont = *-regular,
    BoldFont = *-bold,
    ItalicFont = *-italic,
    BoldItalicFont = *-bolditalic,
  ]%
  \setsansfont{texgyreheros}[
    Extension = .otf,
    UprightFont = *-regular,
    BoldFont = *-bold,
    ItalicFont = *-italic,
    BoldItalicFont = *-bolditalic,
  ]%
  \setmonofont{inconsolata}
}
\newcommand{\swufe@set@cjk@noto}{
  \setCJKmainfont{Noto Serif CJK SC}[
    UprightFont = * Light,
    BoldFont = * Bold,
    ItalicFont = FandolKai-Regular,
    ItalicFeatures = {Extension = .otf},
  ]%
  \setCJKsansfont{Noto Sans CJK SC}[
    BoldFont = * Medium,
  ]%
  \setCJKmonofont{Noto Sans Mono CJK SC}%
}
\newcommand{\swufe@set@cjk@fandol}{%
  \setCJKmainfont{FandolSong}[
    Extension = .otf,
    UprightFont = *-Regular,
    BoldFont = *-Bold,
    ItalicFont = FandolKai-Regular,
  ]%
  \setCJKsansfont{FandolHei}[
    Extension = .otf,
    UprightFont = *-Regular,
    BoldFont = *-Bold,
  ]%
  \setCJKmonofont{FandolFang}[
    Extension = .otf,
    UprightFont = *-Regular,
  ]%
}
\swufe@set@font@pagella
\IfFontExistsTF{Noto Serif CJK SC}{%
  \IfFontExistsTF{Noto Sans CJK SC}{%
    \swufe@set@cjk@noto
  }{\swufe@set@cjk@fandol}%
}{\swufe@set@cjk@fandol}%
%    \end{macrocode}
